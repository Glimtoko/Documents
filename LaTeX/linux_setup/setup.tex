\documentclass[a4paper,10pt]{report}
\usepackage[utf8]{inputenc}
\usepackage{titlesec}
\usepackage{etoolbox}
\usepackage{hyperref}
\usepackage{listings}

\lstset{
basicstyle=\small\ttfamily,
columns=flexible,
breaklines=true
}

\makeatletter
\patchcmd{\chapter}{\if@openright\cleardoublepage\else\clearpage\fi}{}{}{}
\makeatother

\titleformat{\chapter}{\normalfont\huge}{\thechapter.}{20pt}{\huge\bf}

% Title Page
\title{Setup Guide for a Linux Development Environment}
\author{Nick Macey}

\renewcommand{\theparagraph}{\arabic{chapter}.\arabic{paragraph}}
\setcounter{secnumdepth}{4}

\begin{document}
\maketitle


\chapter{Assumptions}
\paragraph{}Libraries, and executables built manually, should be placed in a /prod directory. Access to this should be controlled, but can (and should) be a group distinct to the sudoers.
\paragraph{} The instructions are for a Debian-like disto (Ubuntu, Mint). Broadly, they'll work for any distro, but note that anything using \verb|apt| will be different.
\paragraph{} Assumes that an MPI-aware version of a recent GCC is installed.

\chapter{LLVM}
\paragraph{}LLVM version 11.0 onwards includes the flang (Fortran) compiler.
\paragraph{}Unfortunately, LLVM 11.0 does not exist in any useful pre-built binary or package file.
\paragraph{}Easiest way to build and install LLVM 11.0 is to use the following script:
    \begin{itemize}
     \item \url{https://www.dropbox.com/s/871l5qvebssjack/build_llvm.sh?dl=0}
     \item Will install into /prod/bin, /prod/lib and /prod/include
     \item Note: Building LLVM requires a recent copy of both gcc and gfortran
    \end{itemize}

\chapter{HDF5}
\paragraph{} Pre-built binaries are only available for CentOS – not terribly helpful given that CentOS is dead…
\paragraph{} Download source code from here:
\begin{itemize}
    \item \url{https://www.hdfgroup.org/downloads/hdf5/source-code/}
    \item Ideally, you want the Cmake version (bottom of the page)
 \end{itemize}
\paragraph{} Modify the HDF5options.cmake configuration file (NOT the HDF5config.cmake file)
\begin{itemize}
    \item Switch between MPI and serial builds
    \item Switch on Fortran build
    \item Modify install prefix
 \end{itemize}
\paragraph{} Export compiler executables. E.g. for GCC:
\begin{itemize}
    \item \verb|export CC=gcc CXX=g++ FC=gfortran|
 \end{itemize}
\paragraph{} Build via cmake and make (or cmake --build)
\paragraph{} Note that HDF5 v1.12 does not seem to work with Clang
\paragraph{} Export \verb|HDF5_DIR| as described here: \url{https://portal.hdfgroup.org/display/support/CMake+Scripts+for+Building+Application}

\chapter{SILO}
\paragraph{} SILO is probably the most reliable mesh output format for a general Linux install, although it does suffer from a really bad Fortran API (so we’ll install NetCDF too).
\paragraph{} If you want to use HDF5 with SILO, you’ll need an older HDF5 library (~1.10). But SILO works perfectly well with its own file format.
\paragraph{} Download from here: https://wci.llnl.gov/simulation/computer-codes/silo and follow the build instructions in the INSTALL file.

\chapter{NetCDF}
\paragraph{} Download from here:
\begin{itemize}
    \item \url{https://www.unidata.ucar.edu/downloads/netcdf/}
    \item You want both the C and Fortran libraries, and possibly the C++ one
\end{itemize}
\paragraph{} Build using Cmake, following these guides:
\begin{itemize}
    \item \url{https://www.unidata.ucar.edu/software/netcdf/docs/getting_and_building_netcdf.html#netCDF-CMake}
    \item \url{https://www.unidata.ucar.edu/software/netcdf/docs/building_netcdf_fortran.html}
    \item Remember to set the installation directory to /prod, not the default of /usr/local
\end{itemize}

\chapter{Module Environment}
\paragraph{} Download from \url{https://sourceforge.net/projects/modules/files/Modules/}
\paragraph{} Build instructions are here: \url{https://modules.readthedocs.io/en/stable/INSTALL.html}
\begin{itemize}
    \item You’ll need tclsh installed (\verb|sudo apt install tcl tcl-dev|)
    \item Probably does make sense to install this into /usr/share (see example at bottom of installation guide
    \item Make sure to follow the instructions in the Configuration section in order to have the module environment load
\end{itemize}

\chapter{AMReX}
\paragraph{} Clone AMReX repository: \url{https://github.com/AMReX-Codes/amrex}
\paragraph{} Create BUILD directory
\paragraph{} Run cmake
\begin{itemize}
    \item 
    \begin{lstlisting}
cmake -DAMReX_OMP=YES -DAMReX_FORTRAN_INTERFACES=YES -DAMReX_HDF5=YES -DAMReX_SPACEDIM=2 -DCMAKE_INSTALL_PREFIX=/prod/AMReX2D_CMAKE ..
    \end{lstlisting}
    \item (Requires that \verb|HDF5_ROOT| be set to the root of a parallel HDF5 build)
    \item Probably worth building a debug version too (\verb|-DCMAKE_BUILD_TYPE=Debug|), and maybe both GCC and Clang versions
\end{itemize}


\paragraph{} Make and install
\begin{itemize}
    \item \verb|make; make install|
\end{itemize}


\end{document}          
